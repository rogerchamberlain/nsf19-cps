\documentclass[11pt]{article}
\usepackage{epsfig}
\usepackage{subfigure}
\usepackage{setspace}
\usepackage{url}

\setlength{\textheight}{8.875in} \setlength{\textwidth}{6.5in}
\setlength{\topmargin}{0.0in} \setlength{\headheight}{0.0in}
\setlength{\headsep}{0.0in} \setlength{\oddsidemargin}{0.0in}

\begin{document}
\pagestyle{empty}
\thispagestyle{empty}

\begin{center}
\textbf{\Large Request for REU Supplement}
\end{center}

We propose to involve the undergraduate students in ways that we feel
are beneficial both to the research and to them.  Since the proposed
research involves quite a bit of design and development, the outcomes
are quite tangible, and our past experience indicates that these
tangible outcomes are quite motivational.  Design and development
are things they know how to do when they come in the door, and they
can frequently hit the ground running, only needing to come up to speed
on our particular development environment.

Where they often need to widen the horizon of their experience is the
rigorous quantitative evaluation of an existing design:
how to design experiments, deploy and execute those experiments,
and interpret the results.
To ensure they learn how to do this, we make sure
to incorporate empirical evaluation assignments in their duties.

We have a long record of involving undergraduate students in our research.
Undergraduate co-authoring of papers for us dates back 
over twenty years to the
1990s~\cite{ch94}, and several REU students were supported by the grants
listed in the Results of Prior NSF Support section.

In our lab, advising and mentoring comes from both the faculty and graduate
students. Both are involved in the initial assignment of tasks and
the providing the day-to-day help and assistance that is essential to
enabling a productive learning experience.  The undergraduates are
expected to participate in the weekly research team meetings, both
presenting the work they are doing and contributing to the discussions
of others' presentations.  In short, we treat them like we treat graduate
students, with the only accommodation being that their assigned tasks
are adjusted (relative to that of a graduate student) to be consistent
with their abilities and the time available.

The Computer Science and Engineering Department at Washington University
in St.~Louis has a mature and rich summer REU program. In addition to
the advising and mentorship provided by the PI, REU students participate
in department-wide programs throughout the summer, including:
a technical ``Boot Camp" during the first week of the program,
weekly Research Skills Seminar, weekly Faculty Research Talks,
a Research Symposium during the penultimate week of the program,
and several social events.
As a department, we are committed to the success of our REU students.

The department
advertises each year for students interested in ongoing projects that
span the activities of the entire department.
Recruiting and selection are coordinated at the department level, with
strong PI involvement in the final selection process (described below).

Every year we explicitly recruit students from four year teaching
and liberal arts colleges, women's colleges and predominantly minority
colleges. Target institutions are sent mailings of flyers and brochures
of our program. Department heads and previous year letter-writers are
also contacted via email to advertise the program.
We also advertise our positions to approximately 500 undergraduate
women via the Grace Hopper Celebration of Women in Computing resume database,
which we have access to as Academic Silver Sponsors of the conference.
Each year we receive approximately 75 applications for approximately
15~REU spots.  Students submit a transcript, CV, statement of purpose,
letter of recommendation, and a ranked list of the projects they are
interested in. Students and projects are matched through a coordinated
effort between PIs and the Coordinator of the REU summer program.
Diversity and target institutions are strong but not singular factors
in the matching/admission decisions.

\bibliographystyle{abbrv}
\bibliography{prop}

\end{document}
