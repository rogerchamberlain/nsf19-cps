\section{Evaluation/Experimentation Plan}
\label{sec:eval}

\FIXME{Text from 2018 proposal, needs to be updated.}

In this research we will develop, experiment with, and evaluate three prototype 
catoptric surfaces, each installed in a different environment and each with
unique features that will serve to explore different aspects of our approach.
The first prototype is the system currently being installed in the south window 
of Steinberg Hall's atrium (described in Section~\ref{sec:background}).
The second prototype will be designed for and installed at VelociData, Inc.,
a startup firm in St.~Louis that is located in the recently announced \emph{39~North}
innovation district.
The third prototype will be designed for and installed at BECS Technology, Inc.,
a local manufacturer of electronic control systems that has recently
relocated to a newly redeveloped 42,000~sq.~ft. facility. We now discuss each
installation in more detail, followed by a discussion of our software development
and assessment plans.

\subsection{Steinberg Hall, Washington University in St. Louis}

Steinberg Hall is situated on the Danforth Campus of Washington University
in St.~Louis. It is one of the buildings housing the university's College 
of Architecture, and is within easy walking distance of the Dept. of
Computer Science and Engineering.

The catoptric surface that is being installed at the south end of the
atrium will be complete by the start of the proposed research project.
We will use it for a number of purposes in our proposed research:
\begin{enumerate}

\item \emph{Development and calibration of quantitative daylight delivery models.}
We will evaluate the effectiveness of our current ray-tracing software
system in assessing the impact of different configurations of the surface
(i.e., various mirror positions).  Empirical evaluation will use a number
of light meters distributed within the space. The data collected will be
compared to our predictions as well as the analytical models
of both Bueno et al.~\cite{bwkk15} and Galatioto and Beccali~\cite{gb16}.

\item \emph{Practical aspects.}  We expect to learn a number of
  practical things from this installation. By design, these will
  include the positioning precision that is achievable with our
  current physical design, the viability of operating the mirror
  positioning motors open-loop (the pan-tilt is stepper motor driven
  and the current system does not incorporate shaft encoders or other
  positioning feedback), the benefits (if any) of controlling
  acceleration in addition to position, the timing requirements for
  configuration changes, and the ability to effect multiple mirror
  movements in parallel. In addition, we expect to discover new insights
  beyond these anticipated issues, simply through the direct implementation, 
  experimentation, and evaluation of systems.

\item \emph{Investigation of the ability to provide positioning feedback via
image analysis.}  We will install a camera with the full surface in its
field of view and assess the viability of using image analysis techniques
to discern the orientation of each mirror.  An important component of this
investigation will be to quantify the degree of precision that is achievable.

\item \emph{Quantification of the viable heat transfer.}
We will perform a controlled experiment in which we will focus varying
amounts of sunlight on a vessel of water to determine the temperature
rise that is achievable.  This will enable us to calibrate heat transfer
models that will go into the MDP formulation.

\item \emph{Reliability testing of the components.}
This installation is not permanent. When it is decommissioned, we will
setup a representative collection of the mirror components in our laboratory
for stress testing (to failure).  This will allow us to calibrate our
reliability models for inclusion in the MDPs.

\end{enumerate}

\subsection{VelociData, Inc., 39 North Innovation District}

VelociData, Inc., is located at 10425 Old Olive Street Rd., St.~Louis, MO.
They are in the \emph{39 North} innovation district, which has the Danforth
Plant Science Center, Monsanto, Bio Research \& Development Growth Park,
and Heliz Center Biotech Incubator as anchors. It is a 10~min.~drive from
the Washington University Danforth Campus, and they have given us
permission to install a prototype at their facility (see letter
of collaboration).

There are a pair of potential installation sites at VelociData's location:
an east-facing window provides light to an individual's office, and
(the more interesting option) a pair of west-facing windows that provide
light to a bullpen of cubicles occupied by design engineers.
As we will not be able to tie into the HVAC system at this location, our
objectives will be focused on the daylighting benefits, including their
impact on the occupants of the space\footnote{All experimentation that
includes human subjects will undergo review by the Washington University
IRB prior to implementation.}.

We will use this installation for the following purposes:

\begin{enumerate}

\item \emph{Perform safety testing}\footnote{Note: we only operate the Steinberg
surface under active human supervision, as the automated safe operation cannot yet be 
ensured.}.
We will perform stress testing on the installed system, injecting errors so as to
intentionally attempt to create unsafe conditions, all the while monitoring
to ensure that safety is maintained.

\item \emph{Assess utility of non-southern facing windows.}
Since the windows at VelociData's location are on the top floor, and face
either east or west, this gives us the opportunity to explore the viability
of exploiting multiple-reflection designs
(e.g., a fixed position mirror above the
roofline that redirects sunlight to the active catoptric surface).

\item \emph{Evaluate human responses.}
While we do not have the budget in this project to perform
comprehensive productivity analyses, we will be in a position to
query the building occupants about their experience with the
daylight effects. E.g., do they see value in the controllability
of the quantity of daylight?

\item \emph{Iterate to refine the design.}
As these are prototype systems, we have no expectation that the
initial versions will operate entirely as desired.  We will redesign
and rebuild as needed to make progress on the research questions we
are investigating.

\end{enumerate}

\subsection{BECS Technology, Inc., St. Louis County}

BECS Technology, Inc., is located at
10818~Midwest Industrial Dr., St.~Louis, MO.
They are a small manufacturer of electronic control
systems for a number of markets (e.g., agriculture, aquatics, refrigeration).
The unique benefit to this installation is that they have agreed to
allow us to have access to the HVAC system in their building (see letter
of collaboration).

The HVAC system at BECS is one in which the hydronic water loop
also serves as the fire protection sprinkler system~\cite{Janus01,wm79}.
Individual heat exchangers either deliver heat into the loop 
(e.g., from a boiler) or extract heat from the loop (e.g., to a
cooling tower).
An experimental loop that can be the focus point for light from
a catoptric surface can be readily incorporated into a system such
as this.  The integration is made even easier because BECS, as
a manufacturer of aquatics equipment, uses a controller of its own
design to manage the entire system.

We will use this installation for the following purposes:

\begin{enumerate}

\item \emph{Repeat assessments from initial installations.}
As each installation will have a unique configuration, we will exploit the
latter two installations (at VelociData and BECS) to perform common
experiments at each, comparing the results to increase the confidence
in any conclusions that we draw based on empirical data.
This will include all of the calibration efforts, as well as the comparison
of empirical data to the theoretical models (both for daylight delivery
and thermal heat transfer).
This will also include any human evaluations that are undertaken at
the VelociData site.

\item \emph{Integration into the HVAC system.}
We will implement a heating loop that is capable of delivering
thermal energy into the building's main hydronic water loop.
The temperature of the water in this loop will be logged, and the
master HVAC controller (built by BECS) will enable us to determine
the effectiveness of the heat transfer.  We will also be able to
quantify the amount of energy savings that results.

\item \emph{Assessment of manufacturability.}
As described in the Transfer to Practice Supplementary Document, BECS
is planning to assist us in evaluating the commercial viability of the
catoptric surface as a product.  An important piece of this evaluation is
the assessment (by design and manufacturing engineers at BECS) of the
manufacturability of the system.

\item \emph{Iterate to refine the design.} Again, we do not expect the initial design
to be all that it can be. As for the previous systems, we will redesign
and rebuild as needed.

\end{enumerate}

\subsection{Software Development and Assessment}

The current software for raytracing and physical modeling is within
Rhinoceros 3D (Rhino3D), a free form surface modeling system that utilizes a
non-uniform rational B-spline (NURBS) mathematical representation.
The current optimization (not MDP-based) is coded using
Grasshopper, a visual programming language plug-in.
Grasshopper has been used in the past for lighting performance
analysis~\cite{Echarri16,Willis16}.

The existing positioning software is written in a combination of
Python (executing on a Raspberry Pi) and C (executing on an Arduino Uno).
The Rhino3D/Grasshopper output is made available to the Python code via CSV
files in the filesystem.

The existing MDP model software is written in C++ and currently is not integrated
into the above software components at all.
Clearly, an important software development task in this research project
will be to enable each of the above software elements to interact with one
another, preferably in a relatively seamless manner, aided by modern compilers'
and iterpreters' ability to integrate code written in multiple programming
languages.

When our research group develops software that is intended to support
the research questions being investigated (as opposed to the software
effort being the research question itself), we have a practice of adopting
software engineering techniques that closely approximate those used
in industrial settings.  For example, we have regular code reviews,
testing is performed by individuals who are distinct from the developers,
the software design is presented to the group and its pros and cons
discussed prior to implementation, etc.

The assessment of the software will primarily be embodied in the evaluation
of the catoptric surfaces themselves.  However, we will evaluate the
performance of all of the software, ensuring that execution times are
reasonable for off-line tasks and deadlines are met for real-time tasks.
All software developed in this proposed research will be released as open-source,
and in addition to making the code freely available on the web we will provide
specific mechanisms for the broader research community to interact with our
team to provide feedback, request enhancements, and otherwise engage with
us as we conduct this research.
