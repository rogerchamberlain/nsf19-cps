\section{Project Management and Collaboration Plan}
\label{sec:collab}

\FIXME{Text from 2018 proposal, needs to be updated.}

We first describe the investigator team, including each of our primary roles
in the project.  This is followed by a description of our approach
to collaboration and a timeline for the research activities.

{\bf Roger D. Chamberlain}, PI, is a Professor of Computer Science
and Engineering in the School of Engineering and Applied Science
at Washington University in St.~Louis.
Prof.~Chamberlain will have overall responsibility for managing the
research project and will take the lead in development of Markov
decision process models, performance evaluation, and electrical engineering
design requirements.

{\bf Chandler Ahrens}, Co-PI, is an Associate Professor of Architecture
in the Sam Fox School of Design \& Visual Arts at Washington University in St.~Louis.
Prof.~Ahrens will take the lead in the physical design aspects of
the catoptric surfaces, including their shape, configuration, positioning,
fabrication, and installation.

{\bf Chris Gill}, Co-PI, is a Professor of Computer Science
and Engineering in the School of Engineering and Applied Science
at Washington University in St.~Louis.
Prof.~Gill will lead the software development efforts, with an emphasis
on design for reuse whenever reasonable.  He will also lead our
approach to reusable abstractions that can be generalized to other
cyber-physical system uses.

All three faculty have worked together in different combinations in the 
past, so the organization and management of the present collaboration 
will be straightforward.  Existing publications co-authored by two or 
more of the investigators include~\cite{acmb18,cag18,cagm18,mgc16, mskgct13}.
Ahrens 
and Chamberlain collaborated on the design and implementation of the 
catoptric surface being installed in Steinberg Hall, and Chamberlain 
and Gill have previously co-advised a doctoral student (J.~Meier, 
PhD, currently at Lockheed Martin) who exploited Markov decision 
processes for RF spectrum management.

We will exploit the fact that the entire team is located on the
Danforth Campus of Washington University in St.~Louis to organize
our collaborations around regular (weekly) face-to-face meetings.  
These meetings will form the backbone of the collaboration, where 
we check in with each other to update status, plan next steps, and 
address any issues that have arisen since our previous meeting.
In addition to these regular checkpoints, the investigators and
students involved in the project will meet in different combinations
as appropriate during each week, throughout the conduct of the 
proposed research.

We will also gather for the purpose of reviewing the literature
(traditional journal club activities, which CSE doctoral students
are required to take during their PhD programs), student 
presentations (frequently practice talks for upcoming conference 
and student seminar presentations), and reports to the group from 
anyone who has recently returned from a conference trip.

Our initial plan for the project timeline is listed below, indicating the
primary activities to be pursued in each year of the project.

\subsection*{Year 1}

\begin{itemize}

\item Perform quantitative measurements using the Steinberg catoptric surface.

\item Design initial MDPs, incorporating daylighting and heat harvesting.

\item Develop initial abstractions for low-level controller software.

\item Evaluate imaging approach for positioning feedback.

\item Perform initial design work for VelociData installation.

\item Submit human subjects evaluation plan to IRB for approval.

\end{itemize}

\subsection*{Year 2}

\begin{itemize}

\item Decommission the Steinberg surface and re-purpose components for
reliability testing.

\item Expand MDPs to include additional constraints (i.e., to ensure
safety).

\item Develop initial abstractions for MDP-based optimization software.

\item Install VelociData catoptric surface and perform empirical evaluation.

\item Perform quality of experience evaluation for VelociData users.

\item Perform initial design work for BECS installation.

\end{itemize}

\subsection*{Year 3}

\begin{itemize}

\item Expand MDPs to include additional goals (e.g., to prioritize
reliability).

\item Install BECS catoptric surface and perform empirical evaluation.

\item Perform quality of experience evaluation for BECS users.

\item Refine software abstractions for both low-level positioning control
and high-level MDP-based optimization, based upon what has been learned
from earlier editions.

\item Release software under open source licensing terms.

\end{itemize}
